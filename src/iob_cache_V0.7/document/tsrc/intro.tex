% SPDX-FileCopyrightText: 2024 IObundle
%
% SPDX-License-Identifier: MIT

IOb-Cache is an open-source configurable pipelined memory cache. The
processor-side interface (front-end) uses IObundle's Native Pipelined Interface
(NPI). The memory-side interface (back-end) can also be configured to use NPI or
the widely used AXI4 interface. The address and data widths of the front-end and
back-end are configurable to support multiple user cores and memories. IOb-Cache
is a K-Way Set-Associative cache, where K can vary from 1 (directly mapped) to 8
or more ways, provided the operating frequency after synthesis is
acceptable. IOb-Cache supports the two most common write policies: Write-Through
Not-Allocate and Write-Back Allocate.

IOb-Cache was developed in the scope of João Roque's master's thesis in
Electrical and Computer Engineering at the Instituto Superior Técnico of the
University of Lisbon. The Verilog code works well in IObundle's IOb-SoC system
(https://github.com/IObundle/iob-soc) both in simulation and FPGA. To be used in
an ASIC, it would need to be lint-cleaned and verified more thoroughly by RTL
simulation to achieve 100\% code coverage desirably.
